\documentclass[]{scrartcl}
\newcommand\tab[1][0.7cm]{\hspace*{#1}}
\usepackage{lipsum}
\usepackage{setspace}
\renewcommand{\baselinestretch}{1.5} 


%opening
\title{Criptografia din spatele comer\c{t}ului electronic}
\author{Mazilu George-Viorel, grupa B3}
\date{}
\begin{document}

\maketitle

\begin{abstract}

\end{abstract}

\section*{Abstract}
\tab	Datorit\u{a} dezvolt\u{a}rii tot mai rapide a \textit{Internetului} \c{s}i a dispozitivelor electronice mobile, cump\u{a}r\u{a}turile la distan\c{t}\u{a} devin preferin\c{t}a tot mai multor persoane. Comoditatea de a r\u{a}sfoi magazinele de la distan\c{t}\u{a} a atras popula\c{t}ia s\u{a} aleag\u{a} \textbf{comer\c{t}ul electronic}. Prin \textbf{comer\c{t} electronic} se \^{\i}n\c{t}elege "activitatea de cump\u{a}rare sau v\^{a}nzare prin intermediul transmiterii de date la distan\c{t}\u{a}, activitate specific\u{a} politicii expansive a marketingului companiilor comerciale" [wikipedia]. \^{I}n plus, conform site-ului web "quora", doar 8.3\% din banii existen\c{t}i sunt tip\u{a}ri\c{t}i pe h\^{a}rtie sau monezi, aproximativ $4,3 \cdot 10^{18} $ (triliarde) de dolari americani comparativ cu $51.5 \cdot 10^{18} $ dolari \^{\i}nregistra\c{t}i. \\
\tab Cu toate acestea, pu\c{t}ini cunosc faptul c\u{a} \^{\i}ntregul comer\c{t} electronic nu ar fi fost posibil f\u{a}r\u{a} numeroase principii matematice folosite \^{\i}n criptografia din spatele \^{\i}ntregii activit\u{a}\c{t}i. At\^{a}t siguran\c{t}a informa\c{t}iilor transmise de c\u{a}tre v\^{a}nz\u{a}tori c\^{a}t \c{s}i a banilor tranzac\c{t}iona\c{t}i este asigur\u{a}t\u{a} de c\u{a}tre algoritmi bine pu\c{s}i la punct de c\u{a}tre criptografi \^{\i}ncep\^{a}nd cu anul 1990. Primitivele criptografice precum semn\u{a}turile digitale, func\c{t}iile hash \c{s}i tehnicile de securizare a canalelor de comunicare permit realizarea comer\c{t}ului electronic \^{\i}n condi\c{t}ii de maxim\u{a} corectitudine. De\c{s}i era considerat c\u{a} plata \^{\i}n "bani ghea\c{t}\u{a}" este singurul mod prin care clientul \^{\i}\c{s}i poate p\u{a}stra anonimitatea cump\u{a}r\u{a}turilor, au ap\u{a}rut protocoale electronice variate prin care se pot face tranzac\c{t}ii f\u{a}r\u{a} ca identitatea cump\u{a}r\u{a}torului s\u{a} poat\u{a} fi asociat\u{a} produsului. Cea mai revolu\c{t}ionar\u{a} metod\u{a} de comer\c{t} electronic este tranzac\c{t}ionarea criptomonedelor de tipul \textit{Bitcoin}.  \\
\tab \^{I}n \^{\i}nceputul lucr\u{a}rii de fa\c{t}\u{a} sunt amintite c\^{a}teva din principalele primitive criptografice folosite de la apari\c{t}ia conceptului de comer\c{t} electronic p\^{a}n\u{a} \^{\i}n prezent. \^{I}n continuare, sunt descrise schemele "MicroMint" \c{s}i "Payword" care permit realizarea pl\u{a}\c{t}ilor de valori mici; dup\u{a} care sunt prezentate dou\u{a} scheme cu securitate suficient de ridicat\u{a} pentru realizarea macropl\u{a}\c{t}ilor:  una care ofer\u{a} anonimitate la alegere \c{s}i una care previne \c{s}antajul. \^{I}n plus, se reg\u{a}se\c{s}te o scurt\u{a} descriere a evolu\c{t}iei criptomonedelor "Bitcoin". \^{I}n final, este descris\u{a} sumar tehnologia \textit{"JavaCard"} distribuit\u{a} de compania Oracle \c{s}i implementarea unui protocol folosind aceast\u{a} tehnologie.
\tab  

\end{document}


